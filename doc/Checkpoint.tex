\documentclass[11pt,a4paper]{article} % A4paper option

\usepackage{fullpage}
\usepackage{hyperref}
\usepackage{xcolor}
\usepackage{enumitem} % For more control over lists, helpful for readability
\usepackage{seqsplit} % For breaking long sequences of text

% Defines a command for TODOs to make them stand out in draft
\newcommand{\todo}[1]{\textcolor{red}{\textbf{TODO: #1}}}

\begin{document}

\title{ARMv8 AArch64 Emulator Checkpoint Report}
\author{Team 54}

\maketitle

\section{Group Organisation and Workflow}
\label{sec:group-organisation}

Our group used a structured workflow and leveraged Git with dedicated \texttt{feat/} branches and frequent stub implementations to maintain code quality and compilation. Commit messages followed an imperative format with contributor shortcodes. Every new feature created a GitLab merge request which was reviewed and approved by a separate team member to uphold coding standards.

\subsection{Division of Labour for Emulator (Part I)}
The emulator development was systematically divided into four core components, ensuring balanced responsibilities across the team:

\begin{itemize}[leftmargin=1.5em,noitemsep, itemsep=0.5em, parsep=0em]
    \item \textbf{Jai -- Core System \& I/O:}\\
    Established the emulator's core infrastructure by defining \texttt{ARMState} (registers, PC, PSTATE, memory), initialising its state, implementing the binary file loader, and managing the main execution loop, including command-line parsing, halt instruction, and final output.
    \item \textbf{Richard -- Instruction Decoding \& Bitwise Utilities:}\\
    Led instruction decoding by developing bit extraction utilities, defining C structures (\texttt{\seqsplit{instruction\_types.h}}) for decoded instructions, implementing \texttt{decode\_instruction}, and creating helpers for \texttt{PSTATE} flag updates and register ID mapping.
    \item \textbf{Prasanna -- Data Processing Instructions Execution:}\\
    Implemented the execution logic for all data processing instructions (arithmetic, wide moves, logical, multiplies), developed underlying bitwise shift functions, and integrated \texttt{PSTATE} flag updates.
    \item \textbf{Zayan -- Memory \& Control Flow Instructions Execution:}\\
    Implemented memory access and control flow instructions, including single data transfers (\texttt{ldr}, \texttt{str}) with all addressing modes and load/store size handling, and developed all branch instructions for PC modification, offset calculations, and conditional execution. \todo{For Part III, Zayan will integrate GPIO memory write detection here.}
\end{itemize}

\section{Group Progress and Communication}
\label{sec:group-progress}
\todo{This section will discuss how well the group is working, communication strategies (e.g., daily meetings/ messages), and how you imagine it might need to change for later tasks (e.g., more frequent pairing, clearer ownership on shared components). Keep it to 1-2 paragraphs.}

\section{Emulator Structure and Reusability}
\label{sec:emulator-structure}
\subsection*{Emulator Architecture and Reusability for Assembler}

Our emulator is structured around a standard Fetch-Decode-Execute cycle, operating on a shared \texttt{ARMState} struct that represents the machine's current state. This struct includes an array of 64-bit general-purpose registers (\texttt{registers[31]}), the program counter (\texttt{pc}), a \texttt{PSTATE} struct holding the NZCV flags as bools, and a 2MB byte-addressable memory array.\\

The fetch phase reads 32-bit A64 instructions from memory using the current PC. The decode phase identifies the instruction category using bits 25–28 and extracts relevant fields using bitwise operations, with the help of reusable utility functions (e.g., \texttt{get\_bits()}). A decoded instruction is represented by a \texttt{DecodedInstruction} struct which includes the instruction type as an enum \texttt{type}, a 32-bit \texttt{raw\_instruction} and members for the various instruction-specific fields (e.g., \texttt{dp\_imm\_imm12}). The execute phase performs the instruction logic using helper modules for arithmetic, memory access, or branching, updating the \texttt{ARMState} accordingly and incrementing the PC.\\

Several components of the emulator will be reusable in the assembler, especially the instruction decoding logic and its bit extraction and field parsing logic, which can be reverse engineered to help us generate binary encodings from assembly syntax. Additionally, functions that map register names to IDs and handle 32-bit versus 64-bit register size will be useful when parsing operands.

\section{Challenges and Mitigation}
\label{sec:challenges}
\todo{This section will outline specific technical or collaborative challenges encountered during the emulator's implementation. Examples might include: complex bitwise operations, handling endianness, debugging specific instruction types, or integrating different team members' code. For each challenge, explain your approach to mitigate it. For example, "Understanding ARMv8's shifting rules was challenging, mitigated by extensive diagramming and small-scale test cases." Keep this to 1-2 paragraphs.}

\end{document}