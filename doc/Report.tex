\documentclass[11pt]{article}

% PACKAGES FOR STYLING AND FUNCTIONALITY
\usepackage{fullpage} % Use more of the page
\usepackage{xcolor}   % For colored text, used in the \todo command
\usepackage{amsmath}  % For advanced math environments and symbols
\usepackage{graphicx} % For including images
% \usepackage{listings} % Recommended for formatting code snippets

% CUSTOM COMMANDS
% A \todo command to leave notes for team members
\newcommand{\todo}[1]{\textcolor{red}{\textbf{TODO: #1}}}

% =============================================================================
% DOCUMENT STARTS HERE
% =============================================================================

\begin{document}

% -----------------------------------------------------------------------------
% TITLE PAGE
% -----------------------------------------------------------------------------
\begin{titlepage}
    \centering
    \vspace*{2cm}
    
    {\huge ARMv8 AArch64 – Assembler and Emulator\par}
    
    \vspace{1.5cm}
    
    {\LARGE Final Report\par}
    
    \vspace{0.5cm}
    
    % NOTE: The final report must be a maximum of 6 pages (6 A4 sides).
    % Be concise in your descriptions.
    
    \vfill % Pushes the content below to the bottom
    
    {\Large Team 54\par}
    
    \vspace{1cm}
    
    {\large Date: 20 June 2025\par}
    
    \vspace{1cm}
    
    \textbf{Group Members:}\\
    \begin{large}
        \begin{tabular}{l}
            Richard Baca\\
            Zayan Baig\\
            Prasanna Sivakumar\\
            Jaivir Sohal\\
        \end{tabular}
    \end{large}

    \vspace*{2cm}
    
\end{titlepage}

% -----------------------------------------------------------------------------
% SECTION 1: ASSEMBLER IMPLEMENTATION
% -----------------------------------------------------------------------------
\section{Assembler (Part II) - Design and Implementation}

\subsection{High-Level Design: A Two-Pass Approach}

To handle forward references to labels, we opted for the classic two-pass assembly process, 
simplifying the logic for branch and literal load instructions.

\subsubsection{Pass 1: Symbol Resolution}
In the first pass, the assembler reads the input \texttt{.s} file line-by-line to build a symbol table. 
We maintained a program counter, starting at address zero, which was incremented by four for each instruction or \texttt{.int} 
directive encountered. When a token ending in a colon (\texttt{:}) was found, 
we stripped the colon and added the resulting label along with the address to our symbol table. 

Our data structure of choice for the symbol table was a simple linked list, where each node contained the label,
its corresponding address, and a pointer to the next node. We opted for this data structure due to its simplicity, 
memory and insertion efficiency, and ease of implementation.  


\subsubsection{Pass 2: Code Generation}
The second pass rereads the \texttt{.s} file, this time with a complete symbol table. 
For each line, we invoke the tokenizer to break it into its components. 
A central if-else structure then inspects the mnemonic (the first token) to call the appropriate assembly function 
(e.g., \texttt{assemble\_dp}, \texttt{assemble\_ldr}, \texttt{assemble\_b}). 
The resulting 32-bit instruction word is then written to the output binary file using \texttt{fwrite} in little-endian format.

\subsection{Key Implementation Modules}

\subsubsection{Tokenizer}
Our initial implementation treated chunks like \texttt{[Xn, \#imm]} as single tokens. 
This proved impractical as it made parsing the varied data transfer addressing modes too complex. 
We refactored the tokenizer to treat special characters (\texttt{[}, \texttt{]}, \texttt{,}, \texttt{\#}, \texttt{!}) as individual tokens. 
This change created a more predictable and granular token stream,
which significantly simplified the logic in our data transfer and data processing instruction assemblers.

For the implementation, we used a simple resizing array for memory efficiency.
We first skip any whitespace, then save the start of the token, and finally continue until we hit a whitespace or special character.
Then we extract the token based on the start and current pointer and add it to the array.

\subsubsection{Data Processing (DP) Instructions}
To handle the large number of DP mnemonics, we implemented a lookup table that maps each mnemonic to one of ten DP mnemonic "types" 
(e.g., \texttt{ARITH}, \texttt{LOGICAL}, \texttt{MOVX}, etc.), based on the categorization found in Table 2 of the spec.
For each type, we wrote a dedicated assembly function that handles the specific encoding logic. 
This modular approach allowed us to easily handle mnemonic aliases, for example, the instruction \texttt{mul rd, rn, rm} is 
assembled by passing the appropriate tokens to the assembly function for instructions of type \texttt{madd rd, rn, rm, rzr}.

We also implemented a number of helper utility functions such as \texttt{set\_bits} to make the encoding logic cleaner and more maintainable
and \texttt{get\_register\_number} to avoid duplicate code when converting register names to their corresponding numbers.

We encountered a challenge of many tests failing which we traced back to the handling of tokens by the DP assembler. 
The initial implementation assumed the first, less granular version of the tokenizer.
After refactoring the DP assembler to work with the correct tokenizer, we fixed around 570 DP-related test cases.

\subsubsection{Data Transfer (LDR/STR) Instructions}
The data transfer instructions were the most complex to implement due to their varied and flexible addressing modes. Our improved tokenizer was essential here. By examining the pattern of tokens - such as the presence of an exclamation mark for pre-indexing, or the number of tokens between the square brackets - we could reliably distinguish between the different addressing formats and call the correct assembly logic.
For example, to identify the unsigned immediate offset addressing mode, we looking for the presence of a hashtag '\#' within a pair of brackets. We could then confirm this was not a pre-indexed address due to the lack of an exclamation mark at the end of the address.

\subsubsection{Branch Instructions}
The core logic for assembling branch instructions is the calculation of the branch offset from a label. The offset is calculated using the formula: \texttt{offset = (target\_address - current\_instruction\_address) / 4}. We ensured our implementation correctly handled this as a signed, 4-byte-aligned, word-based offset. The target address was retrieved from the symbol table built in the first pass. 
The C code correctly handles sign extension for the \texttt{simm19} (B.cond) and \texttt{simm26} (B) fields implicitly. When the word offset is calculated, if it's negative, the resulting \texttt{int32\_t} value naturally holds its two's complement form. Subsequent bitwise ANDing with the field-specific mask (e.g., \texttt{0x03FFFFFF} for \texttt{simm26}) correctly truncates this to the required bit width while preserving the sign for negative offsets.

% -----------------------------------------------------------------------------
% SECTION 2: RASPBERRY PI IMPLEMENTATION
% -----------------------------------------------------------------------------
\section{Raspberry Pi (Part III)}

\subsection{Assembly Program Design (led\_blink.s)}
Our assembly program, \texttt{led\_blink.s} used to blink an LED on the Raspberry Pi works as follows:
\begin{itemize}
    \item Load the base address of the GPIO peripheral registers into a register.
    \item Configure the target GPIO pin (e.g., GPIO 21) as an output by writing the bit pattern \texttt{001} to the appropriate bits in the GPFSEL2 register.
    \item Enter an infinite loop labelled \texttt{main\_loop:}.
    \item Turn the LED on by writing to the GPSET0 register.
    \item Implement a delay using a nested loop that decrements a register a large number of times.
    \item Turn the LED off by writing to the GPCLR0 register.
    \item Implement another delay.
    \item Use an unconditional branch (\texttt{b main\_loop}) to repeat the cycle indefinitely.
\end{itemize}

\subsection{Challenges}
Our emulator and assembler both passed all tests, but we faced challenges when deploying the assembly program to the Raspberry Pi.
The main issue was that connecting the SD card containing the assembled 
\texttt{kernel8.img} file did not flash the LED as expected.
At first, we thought the issue was with the hardware, but after connecting the LED to the steady 5V pin, 
we confirmed that the LED and the GPIO pin were functioning correctly.
In the end, we discovered that we were connecting the SD card to the Raspberry Pi incorrectly;
we were plugging in the USB card reader into a USB port instead of the SD card itself into the SD card slot.
After correcting this, the LED started blinking as expected.

% -----------------------------------------------------------------------------
% SECTION 3: THE EXTENSION
% -----------------------------------------------------------------------------
\section{The Extension}

\subsection{Description and Goal}
Our extension, the "Guide Glove," aims to assist visually impaired individuals in navigating complex or confined spaces where a traditional cane might be cumbersome or less effective. By integrating an ultrasonic sensor, a buzzer, and a vibration motor with a Raspberry Pi, the glove provides real-time, multi-modal feedback about the proximity of nearby objects. The ultrasonic sensor measures the distance to surfaces in front of the hand, and this data is translated into varying intensities of haptic (vibration) and auditory (buzzer tone/frequency) feedback, which intensify as an object gets closer, thereby enhancing spatial awareness and user safety.

\subsection{Design and Implementation}
The "Guide Glove" extension was realized as a C application on the Raspberry Pi, interfacing with an HC-SR04 ultrasonic sensor, a piezo buzzer, and a vibration motor, initially prototyped using a breadboard and necessary resistors. The program utilized the \texttt{pigpio} library for GPIO control.

Core logic involved continuously reading a distance value (in cm, from a file populated by the sensor process) and translating this into synchronized feedback. Auditory feedback featured a dynamically pitched buzzer and a rhythmic beeping where the silent interval decreased with proximity. Haptic feedback mirrored this rhythm with the vibration motor. A critical "danger zone" (5cm) triggered constant, high-intensity sound and vibration. If no object was in range, outputs were silenced.

\subsection{Example of Use}
The Guide Glove serves visually impaired users in two primary scenarios. Firstly, for general navigation in unfamiliar environments, such as a new restaurant, the user can orient the glove (e.g., sensor facing forward from the hand or near the head) to receive haptic and auditory warnings about approaching obstacles like walls or low-hanging objects, preventing collisions. Secondly, for close-proximity tasks, the user can employ a sweeping motion with the glove to locate specific items within a confined space, for instance, finding a mug in a cupboard without accidentally knocking over other objects, guided by the intensifying feedback.

\subsection{Challenges and Learning}
The primary challenge was iterative hardware refinement. Initial breadboard connections proved unreliable, prompting a move to soldered components on perfboard. This transition introduced new issues like wire breakages and misconnections, which were time-consuming to debug. Our key learning was the necessity of rigorous, incremental testing after each hardware modification to quickly isolate faults, which significantly improved our debugging efficiency and the final prototype's robustness.


% -----------------------------------------------------------------------------
% SECTION 4: TESTING STRATEGY
% -----------------------------------------------------------------------------
\section{Testing Strategy}

\subsection{Assembler Testing}
\todo{Describe how you tested the assembler. Did you use the provided test suite? Did you write your own unit tests or assembly files? Mention specific cases you focused on (e.g., forward/backward branches, all addressing modes, instruction aliases, edge cases).}

\subsection{Raspberry Pi and Extension Testing}
For our C-based "Guide Glove" extension, testing was performed directly on the Pi. After compiling with \texttt{gcc -lpigpio -lrt -lpthread}, we ran the executable and systematically presented objects at varying distances to the ultrasonic sensor. We verified: (1) accurate distance reporting to the terminal, (2) corresponding activation of the vibration motor, and (3) correct dynamic pitch and rhythm changes from the buzzer, including the distinct "danger zone" feedback. Iterative testing after hardware modifications (soldering, pin connections) was crucial to ensure component reliability.

\subsection{Effectiveness of Testing}
Our multi-faceted testing strategy, combining the provided test suite with custom-written assembly files for edge cases (e.g., negative branch offsets, zero-value immediates, instruction aliases), proved highly effective. The suite ensured specification compliance, while our targeted tests helped isolate specific issues. The hardware deployment for Part III served as a crucial end-to-end validation, leading to emulator enhancements for MMIO logging that improved our debugging capabilities for hardware-specific code. This comprehensive approach allowed us to identify and resolve most issues systematically.


% -----------------------------------------------------------------------------
% SECTION 5: GROUP REFLECTION
% -----------------------------------------------------------------------------
\section{Group Reflection}
\todo{Write a reflection on your experience working as a group. Discuss your methods for communication (e.g., Discord, regular meetings) and for splitting up work (e.g., by feature, pair programming). Critically evaluate how effective these methods were. What would you keep the same for a future group project, and what would you do differently?}


% -----------------------------------------------------------------------------
% SECTION 6: INDIVIDUAL REFLECTIONS
% -----------------------------------------------------------------------------
\section{Individual Reflections}
\subsection{Reflection from Richard}
I feel like I made solid contribution to our project, especially in fixing most of the failing test cases and implementing the 
instruction decoding and data processing assembly. I also spent a fair bit of time reviewing merge requests, which proved tricky 
when the branches were large but I tried to stay as thorough as possible. Overall, I think my strengths lied in problem-solving and 
consistency. If I were to do anything differently, it would be to allocate more time to complex code 
reviews to avoid potential oversights. I've learned a lot from this project and will definitely carry that forward into future group work.

\subsection{Reflection from Zayan}
My main contributions to the group were through my sections of the code. For both our emulator (Part I) and assembler (Part II), I implemented the 
execution and parsing of the data processing instructions. I also wrote the high level code for our assembler to link everyone's work individual work 
together so that our assembler could actually run. It was noted that my biggest strength was being thorough and getting "it" right first time, meaning that 
my work needed minimal changes during code reviews and testing. One thing that I have learned is that I need to ensure that I keep closer contact with 
team members when writing code - to ensure that we don't accidentally overlap and write the same code. This happened once in Part I, and we learned from our 
mistake and it didn't occur again.

\subsection{Reflection from Prasanna}
My primary contributions included implementing the DP executor and register shifting for the emulator (Part 1), the branching logic for the assembler (Part 2), and contributing to the hardware implementation and testing for our Part 4 extension. I consistently aimed to produce well-commented, functional code, particularly for the assembler's branching. Initially, I recognized a weakness in my proactivity regarding group communication, testing, and peer reviews. However, I consciously worked to improve this, becoming more engaged during the extension phase. This project reinforced the value of rigorous Git practices and highlighted the importance of consistent team communication to prevent conflicts—a key area I will focus on more in future group work.

\subsection{Reflection from Jai}
As team leader, my role involved both project coordination and technical development. I was responsible for structuring our workflow by breaking down the 
main emulator and assembler tasks into manageable components for the team. My focus for the first two parts was to build the stable software foundation that 
everyone else would depend on. This included implementing the central ARMState struct for the emulator and developing the two most critical utilities for the 
assembler: a tokenizer and the hash-map based symbol table. Creating a clear pipeline for these modules was essential as they were on the project's main path, 
enabling the team to work in parallel with minimal integration issues.


For our extension, I was more involved with hardware and software co-design, writing the C control system for the sensor glove. My primary contribution here was 
designing and tuning the non-linear feedback algorithm. I felt a simple linear response would be uninformative, so I developed a system that uses both variable 
pitch and rhythm to create more intuitive user feedback. This was implemented using the pigpio C library to generate precise audio tones. I also worked with the 
team on the physical prototyping, which included soldering the transistor circuits required to safely drive the motors from the GPIO pins.

I believe my main strength was in creating these core systems, which helped structure our team's collaboration. The biggest challenge was moving from the 
predictable software environment of the emulator to debugging the physical hardware, where a bug could be in the code, the wiring, or even the network 
connection for SSH. This experience taught me the value of methodical, incremental testing, such as using a Python script to quickly validate a hardware 
connection before committing to a full C implementation. Overall, the project provided a valuable view of systems engineering.
\end{document}